\begin{abstract}{Abstract}
The equity risk premium puzzle - even its very existence - is arguably one of the most fascinating as well as persistent phenomenons in the field of macro finance that has been attempted to rationalise since the famous seminal paper by \citet{Mehra1985}.\\
Following the disaster risk approach initially proposed by \citet{Rietz1988} and further developed by \citet{Barro2006} I conduct a comprehensive study of asset returns, macroeconomic aggregates and moments related to disastrous events covering almost 150 years of financial markets and economic history across 16 developed countries.\\
Whereas previous studies of disaster risk focused on the reconciliation of economic theory and empirical data using a) aggregate output and b) rationalising the puzzle at a global level my contribution is twofold:

Firstly, I extend the analysis to private consumption expenditure data, a more accurate measure for consumption-based asset pricing for a now-available greater time horizon and secondly I allow for cross-country heterogeneity in characteristics with respect to disaster propensity/experience.

The major finding of my dissertation is that accounting for the effect of disasters on consumption risk (and output risk) does indeed rationalise the puzzle and country-specific preference parameter estimates are presented along with plausible model predictions for the rates of return on equity and risk-free, short-term government bills which appears to attenuate the equity premium puzzle if it also addresses the risk-free rate puzzle \citep{Weil1989} jointly.
\end{abstract}

